\documentclass[12pt, a4paper]{article}
\title{Shame - Rough Draft}
\author{Dylan Dunn}
\date{July 2018}
\begin{document}

\maketitle

\begin{center}
    \emph{What do you consider the most humane? - To spare someone shame. What is the seal of liberation? - To no longer be ashamed in front of oneself.}
    \rightline{- Friedrich Nietsche}
\end{center}

\begin{abstract}
Shame is a complex subject to navigate both in life and academically. 
It creeps into every area of one's life. 
Not neccessarily always in an explicit and boastful manner, but sometimes quietly. 
Having subtle effects on all that it touches. This paper explores both why we shame
and when it should be listened to, if at all. 
\end{abstract}

\paragraph{What is it?}
The act of being shamed or feeling ashamed is simple enough to describe and define. 
One feels a sense of regret and embarrassment about one's actions. And this is the ultimate goal of the shamers.
To discourage an individual or group from engaging in some behaviour. 
However, the more interesting question here is ``why?''

\paragraph{Why do we do it?}
To answer the question of whether we should pay attention to those who try and shame us we first must determine what 
they are trying to achieve. 
It is important to note here that shame is only the end result. The reasons we do it are a much broader and complex matter.

The first reason I'd like to discuss is that shame, potentially, is a means to enforce to social norms.
By discouraging behaviours that violates norms we ensure that all indivuduals in a society act in accordance
with certain "laws". I think this is certaintly part of the equation. There are a vast number of actions that, 
while not always "illegal", are taboo. Homosexuality, drug use, and abortion are a few among many taboos that aren't 
always discouraged at a judicial level, but are discouraged by individuals of a society. One argument of the shamers
in this situations is that they are discouraging behaviours which are in some way sinful. They are the enforcers of moral action
even if that there is no pragmatic reasoning behind doing so. But, isn't that the job of the state? Do these shamers always represent the ideals and beliefs of a society.
The goal of a democracy is indeed to uphold the beliefs of a society as a whole. Why not fight the law? Why not fight to make what you find
taboo illegal? The answer is simple, the majority will respond to shame. What most of us don't do, is shaped by this. 
This leads to an overarching answer to what this style of shame aims to achieve, and its about control. You can control other actions
and ensure that they are inline withy your personal beliefs through shame. 

For the sake of hyperbole lets say you wanted to stop someone from drinking milk. There is no legal reason why they can't, you can 
convince them to heed such a mundane action. 

I think Nietsche makes a damn near inarguable theory on the purpose of shame in his Geneoalogy of Morals. His master/slave 
explanation works too well not to have at least some aspects of truth. One uses shames as a non-objective, manipulative way to 
convince someone they are wrong. It's a great sword against any rational argument, shame. 
However, there seems to be a shield against this mad reasoning, independence, stoicism, and confidence
can defeat any amount of shame. 



\paragraph{Should we listen to it? }
If shame is just a means of getting power then is it something worth listening to? I'd like to say no, but why am I met 
with such resistance. Would it really be wise to disregard the stink eye of passive observers? To go and do whatever you wish?
No, of course not! One should not stroll nude in the work place, not because he should be ashamed to do so, but because
pragmaticically speaking he'd lose his job, clients, friends, etc. One should not partake in this activity based
on his own judgement not the judgement of others. 

We should not stray from activities just because others find them wrong. One should use sound reasoning to make these decisions, 
if one is not confident in their own abilities they should seek the advice of others. We should listen to those who seek to help
through advice liken to "You shouldn't do that because I don't think it's potential consequences are worth it". What advice is
"You shouldn't do that because it's wrong, everyone who does that should think badly of themselves". 

I'm not exluding the non-pragmatic, one can make judgments based on subjective beliefs, but the important part is making those
final judgments in their own heart. 

\end{document}