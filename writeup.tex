\documentclass[12pt, a4paper]{article}
\usepackage[utf8]{inputenc}
\usepackage{chemfig}
\usepackage{textcomp}
\usepackage{siunitx}
\title{Extraction of Psychoactive Alkaloids from P. arundinacea}
\author{Dylan Dunn}
\date{July 2018}
\begin{document}
%\tableofcontents{}
\maketitle

\begin{abstract}
    An attempt to see if psychoactive compounds can be efficiently extracted from a limitless and easily attainable plant.
    This report details the proccess in which is this was achieved. The purpose of developing this process is to .
    More importantly, this process was carried out to provide more concrete evidence that P. arundicacea is in fact a source of psychoactive alkaloids in light of adundant speculation.
\end{abstract}

\begin{figure}[h]
\centering
\chemfig{N(-[:330])(-[:90])(-[:210](-[:150]-[:210]*5(-(*6(-=-=-=1))--{HN}-=)))}
\caption{N,N-Dimethyltryptamine}
\end{figure}

\section{Introduction}
    P. arundicacea also known as reed canary grass is a common grass widely distributed in Europe, Asia, northern Africa and North America\cite{USDA15}.
    A variety of pschoactive compounds have been found in its leaves, most notably N,N-DMT, 5-MeO-DMT, and bufotenin\cite{Smith76}.
    For centuries these compounds have been consumed orally by indigenous people of the Amazon basin through a brew of exotic plants called ayahuasca.
    In the past couple decades these plants namely \emph{Pschotria virdis} and \emph{Mimosa hostilis} have been used to create a pure crystallized form of the drug known as DMT.
    However, plants traditionally used as a source of DMT alkaloids, such as those mentioned, are not widespread and are often imported at high cost.
    Due to P. arundicacea's relative abundance if a method of extracting these alkaloids could be developed and refined it would provide just about anyone with an easily attainable and abundant source of DMT at little to no cost. 
    
    Reed canary grass provides some challenges as compared to other DMT sources. The plant contains a lot of other organic matter which must be removed. 
    As we are targeting the leaves and not the bark or root we will have to remove a large amount of organic matter from the solution before crystallization.
    Additionally, reed canary grass purportedly averages alkaloid concentrations significantly lower than that of previously mentioned sources.   
    Several attempts were made and the processed refined. The following discusses the process which resulted in a near pure isolation of the target alkaloids.
\section{Materials and Methods}

\subsection{Materials}
\begin{itemize}
    \item 250ml Naptha
    \item 150ml 5\% Acetic Acid (Distilled White Vinegar)
    \item 125g P. arundicacea Leaves
    \item 5g Sodium Hydroxide
    \item 500ml Distilled Water
    \item 500ml Seperation funnel
    \item 1000ml Erlenmeyer flask
    \item Safety Goggles and Safety Equipment
    \item Laboratory Funnel
    \item 100ml Graduated Cylinder
    \item Ring Stand and Clamp
    \item PH Meter or Paper
    \item 2x 500ml Beaker
    \item Cotton Balls 
    \item Filter Paper
\end{itemize}

\subsection{Methods}

\paragraph{Collection and Preparation of P. arundicacea.}
    Leaves were collected in late spring when the seedheads were violet in hue (Flowering). The blades were removed at the collar and only green, unblemished leaves were kept.
    The leaves were then layed lengthwise and cut into 1" strips. 125g were measured out and placed into a sealed plastic bag and the air removed. The filled bag was then froze.
    24hrs later the bag was removed and allowed to thaw at room temperature. This was repeated for 2 additional cycles for a total of 3 freeze/thaw cycles.
    This results in cell lysis which releases our target alkaloids into solution. On the last cycle the leaves, still frozen, were placed into a food processor and pulverized into a fine powder. 
    200ml of water were added and the solution was processed until a slurry formed. The solution was then transfered into a fine mesh strainer above a large beaker.
    Once the majority of the liquid had drained through the remaining fibers were wringed above the strainer to remove any remaining liquid.
    The fibers were then discarded leaving a green solution. 
    To remove the remaining fiber particulate a cotton ball was placed in the bottom of a laboratory funnel and the solution filtered through into another beaker.
    The filter clogged halfway through and the cotton had to be replaced. This left a green solution with little to no large fiberous particulate.

\paragraph{Isolation of alkaloids.}
    With the majority of physical contaminates removed we can begin to isolate our target alkaloids from solution. 
    A basic solution was prepared by the addition of 5g of NaOH to a graduated cylinder containing 100ml of distilled water set in an ice bath.
    The NaOH solution was added slowly to our grass solution with strong stirring until the PH of the grass solution read 12.
    50ml of naptha were measured in a graduated cylinder and a seperation funnel was setup in a ring stand above a beaker. 
    The basified solution was decanted into the top of the seperation funnel followed by the measured solvent. 
    The seperation funnel was then capped, removed from the clasp, and moderately agitated. Care was taken to prevent an emulsion from forming.
    The seperation funnel was then reclamped and allowed to rest for approxamately 5 minutes.
    
    With the layers seperated the lower portion was slowly removed leaving a few ml in the funnel so as to have no contamination from the naptha. The napatha solution was then discarded.
    A 5\% solution of acetic acid was prepared, in this case distilled white vinegar was used. The acetic acid was slowly added to our basic solution with stirring until a PH of 4 was read. 
    Our target alkaloids should now be soluble in our non-polar solvent. Our acidified solution was then decanted into a seperation funnel followed by 100ml of napatha.
    The funnel was capped and shaken gently in order to prevent an emulsion from forming. The layers were then allowed to seperate for 10 minutes. 
    A cloudy green upper layer formed above a dark green/brown bottom layer. The bottom layer was then removed in its entirety making sure only the naptha solution remained.
    The top layer appeared to be an emulsion. To break it a cotton filter was prepared as described previously and the solution filtered through. 
    The filter was successful in breaking the solution into a green lower layer and a yellow solvent upper layer. 
    The seperation funnel was then used to remove all of the lower green layer, which was discarded. The upper layer was saved for purification. 
    

\paragraph{Purification and Crystallization of Alkaloids.}
    The yellow discoloration is not from our target alkaloids. While it doesn't need to be removed, doing so will leave us with purer crystals. 
    In order to remove it activated charcoal was chosen. A cotton filter was prepared and activated charcoal was used to pack the lower third of the filter. 
    The solution was then decanted through the filter.
     The filter was successful in removing the contaminates causing the yellow discoration leaving a crystal clear solution with fine black particulate at the bottom.
     The dark particulate was dust from the activated charcoal used. To remove it the solution was passed through filter paper.
     The clear solution was then evaporate at room temperature over 72 hours until only 20ml of solvent remained.

\section{Results}

\paragraph{Yield of target alkaloids.} Following crystallization the final wieght of the material was weighed at $\SI{50}{\mg} \pm \SI{5}\mg$. 
    Given 125g of starting material this gives us an alkaloid content of $\SI{0.0004}\percent \pm \SI{0.00004}\percent $. 
    This is rather low compared to other sources, however this is not the percentage of dry weight material.
    Grass used for hay averages a moisture content of 65\%. With this we can assume that 2/3 of our starting material was water 

\begin{table}[h]
\begin{center}
\begin{tabular}{|c | c|}
    \hline
    DMT Source & Alkaloid Concentration \\
    \hline
    Acacia acuminata & 1.5\% \cite{lycaeumacacia}\\ 
    Psychotria viridis & 0.1-0.61\% \cite{amazingnature}\\
    Mimosa hostilis & 0.31-0.57\% \cite{erowid}\\
    Phalaris arundicacea & 0.0004\%\\
    \hline
    Our yield & 0.0004\% \\
    \hline
\end{tabular}
\caption{Comparison of alkaloid concentration to other DMT sources}
\end{center}
\end{table}

\section{Discussion}

\begin{thebibliography}{9}
    \bibitem{USDA15}
        Agricultural Research Service(ARS), United States Department of Agriculture
        \textit{"Phalaris arundicacea" Germplasm Resources Information Network (GRIN)}.
    \bibitem{Smith76}
        Terence A. Smith
        \textit{Tryptamine and Related Compounds in Plants}. Phytochemistry, 171-175, 1977.
    \bibitem{mack88}
        Joseph P. G. Mack, Dawn P. Mulvena, and Micheal Slaytor
        \textit{N,N-Dimethyltryptamine Production in Phalaris aquatica Seedlings}. Plant Physiol, 315-320, 1988.
    \bibitem{lycaeumacacia}
        leda.lycaeum.org
        \textit{"lycaeum \textgreater Leda \textgreater Acacia acuminata"}, 2001.
    \bibitem{amazingnature}
        Amazing-nature.com
        \textit{"Amazing Nature \textgreater Pschotria Viridis"}, 2007.
    \bibitem{erowid}
        Erowid.org
        \textit{"Ask Erowid : ID 75 : What is the DMT content of Mimosa hostilis rootbark?"}, 2014.
    \bibitem{}
\end{thebibliography}

\end{document}